%Curriculum Vitae

\documentclass[12pt,a4paper]{article}
\usepackage{fancyhdr,url,soul,sectsty,hanging}
\usepackage[a4paper, top = .85in, bottom = .85in, left = .75in, right = .75in]{geometry}
\usepackage[parfill]{parskip}
\renewcommand{\familydefault}{\sfdefault}
\fancyfoot{}
\fancyhead{}
\pagestyle{fancy}
\fancyhead[L]{}
\fancyhead[C]{\textbf{\huge{Douglas Keller Jr.\\}}}
\fancyhead[R]{}
\fancyfoot[L]{}
\renewcommand{\headrulewidth}{0.0pt}
\allsectionsfont{\normalsize{}\MakeUppercase}
\sectionfont{\normalsize{}\underline{}\MakeUppercase}

\begin{document}

\section*{Contact}

\textbf{Email:} dg.kllr.jr@gmail.com; dkeller12@alaska.edu\\
\textbf{Phone:} 1-907-342-2070\\
\textbf{Website:} www.alaskanresearcher.org

\section*{Education}

\textbf{Master’s of Science: Mechanical Engineering}\\
\textbf{University of Alaska Fairbanks}, \textit{Fairbanks, Alaska, USA}
\begin{itemize}
\item[] \textit{Fall 2017 – Fall 2018}
\item[]\textit{GPA: 4.0/4.0}
\end{itemize}

\textbf{Bachelor’s of Science: Mechanical Engineering}\\
\textbf{University of Alaska Fairbanks}, \textit{Fairbanks, Alaska, USA}
\begin{itemize}
\item[] \textit{Fall 2014 – Fall 2018}
\item[] \textit{GPA: 3.83/4.0}
\end{itemize}


\section*{Research}

\textbf{Master of Science Research Thesis} \textit{Fall 2017 – Fall 2018}\\
\textbf{College of Engineering and Mines, University of Alaska Fairbanks}, \textit{Fairbanks, Alaska, USA}\\
\textbf{NASA Armstrong Flight Research Center}, \textit{Edwards Air Force Base, California, USA}
\begin{itemize}
\item Using NASA Fiber Optic Sensing System (FOSS) and conventional strain gauge systems in high electromagnetic interference environments found in aviation
\item Developed a load cell designed around the FOSS to measure thrust from an electric motor coupled with a 3-blade propeller to develop future electric propulsion systems
\item Analyze and compare data retrieved using FOSS and conventional strain gauges
\item Determined the effect of electromagnetic interference on the thrust loading measurements
\item Determined the effect of mechanical interference/vibration with both systems in same
application
\end{itemize}

\textbf{Alaska Space Grant Program Undergraduate Research Fellowship} \textit{Fall 2017 – Summer 2018}\\
\textbf{Geophysical Institute, University of Alaska Fairbanks}, \textit{Fairbanks, Alaska, USA}
\begin{itemize}
\item Using NASA MPLNET and Radiosonde data to determine the intraday variability of the high
latitude atmospheric boundary layer
\item Developing algorithms to determine atmospheric boundary layer structural parameters from Micro Pulse Laser Radars (Lidar)
\item Compare and contrast collocated Lidar and Radiosonde retrievals of ABL parameters
\item Produce statistical estimates of ABL parameters and compute diurnal and seasonal
variations
\item Analysis of high latitude ABL with MPLNET, RAOBS, and GPSRO
\end{itemize}

\section*{Publications}

\subsection*{Submitted}

\begin{hangparas}{20pt}{1}
\underline{D. Keller}, D. R. Eagan, G. J. Fochesatto, R. Peterson. “Advantages of Fiber Bragg Gratings over
Resistance-Based Strain Gauges in the Presence of Electromagnetic Interference Emitted from
an Electric Motor for Aerospace Application.”
\end{hangparas}

\subsection*{In Progress}

\begin{hangparas}{20pt}{1}
\underline{D. Keller}, G. J. Fochesatto. “A New Wavelet to Determine the Planetary Boundary Layer Height
from Micro Pulse Lidar Backscatter.”
\end{hangparas}

\subsection*{Conference Proceedings}

\begin{hangparas}{20pt}{1}
Fochesatto G. J., O. Galvez, P. Ristori, \underline{D. Keller}, and E. L. Fochesatto. “Lidar to Determine the
Fractions of Ice, Liquid and Water Vapor in Polar Tropospheric Cloud.” \textit{Proceedings of the 28th
International Laser Radar Conference}, Bucharest, Romania. 25-30 June 2017.
\end{hangparas}

\subsection*{Conference Poster Presentations}

\begin{hangparas}{20pt}{1}
\underline{D. Keller}, G. J. Fochesatto. “RAOBs and Micro Pulse Lidar Determination of the Atmospheric
Boundary Layer.” \textit{Alaska Space Grant Program Annual Symposium}, Anchorage, Alaska, USA.
April 20, 2018.

\underline{D. Keller}, D. Eagan. “FOSS Load Cell.” \textit{Undergraduate Research and Scholarly Activity Research Day}, Fairbanks, Alaska, USA. April 10, 2018.

\underline{D. Keller}, G. J. Fochesatto, Ellsworth Welton, Jasper Lewis, James Campbell, and Sebastian Stewart. "Methodology for PBL Retrieval Based on NASA MPLNET Datasets." \textit{99th Annual American Meteorological Society Meeting}, Phoenix, Arizona, USA, Jan. 6-10, 2019.
\end{hangparas}

\section*{Work Experience}

\textbf{Temporary Research Technician} \textit{Spring 2019 - Present}\\
\textbf{Alaska Center for Energy and Power, University of Alaska Fairbanks}, \textit{Fairbanks, Alaska, USA}
\begin{itemize}
\item Rewriting the Alaska Center for Energy and Power's (ACEP) Energy Technology Facility's (ETF) safety manual
\item Determining the job hazards and required training to safely perform experiments in the ETF
\item Adjusting the emergency action plan of ETF to more effectively transfer ETF's emergency systems
\end{itemize}
\textbf{NASA Mechanical Engineering Intern} \textit{Summer 2017}\\
(funded by the Alaska Space Grant Program)\\
\textbf{NASA Armstrong Flight Research Center}, \textit{Edwards Air Force Base, California, USA}
\begin{itemize}
\item Tested thermodynamics and heat transfer of the initial Fiber Optic Sensing System (FOSS)
enclosure concept for the Quiet Supersonic Technology (QueSST) X-Plane (now the X-59)
\item Researched heat transfer technologies for FOSS enclosure flight testing application such
as heat pipes and thermoelectric coolers
\item Analyzed heat transfer methods for FOSS enclosure application including foam insulation
\item Designed prototype enclosure for FOSS components, utilizing analyzed methods; features
foam insulation and heat pipes for improved high temperature environment survivability
\end{itemize}

\textbf{Engineering Intern} \textit{Summer 2015}\\
\textbf{Alaska Department of Transportation}, \textit{Fairbanks, Alaska, USA}
\begin{itemize}
\item Measured state maintained pedestrian facilities such as sidewalks and ramps
\item Determined pedestrian facility requirements for American Disability Act compliance
including required slope accommodation for disabled pedestrians
\item Compiled measurement data from multiple interns for coordinator review
\end{itemize}

\section*{Professional Projects}

\textbf{Raman Spectroscopy Lidar Project} \textit{Fall 2016}\\
(funded by the National Science Foundation)\\
\textbf{Geophysical Institute, University of Alaska Fairbanks}, \textit{Fairbanks, Alaska, USA}
\begin{itemize}
\item Gained experience in optomechanical design for Lidar development
\item Developed simulation of Lidar signals to condition instrument design for 532 nm, N2, and
H2O Raman vibrational bands
\item Implemented multichannel Lidar receiver
\end{itemize}

\textbf{Golf Swing Replicator Prototype} \textit{Summer 2016}\\
\textbf{Project Aisle}, \textit{Spokane, Washington, USA}
\begin{itemize}
\item Designed the golf swing replicator prototype in SolidWorks 2014 with adjustable stand, ball
holder, and swinging mechanism
\item Fabricated the golf swing replicator prototype, utilizing the mill, lathe, and MIG welder
\item Tested and analyzed golf swing replicator prototype
\end{itemize}

\textbf{Ice Arch Build} \textit{Fall 2014 – Spring 2015}\\
\textbf{College of Engineering and Mines, University of Alaska Fairbanks}, \textit{Fairbanks, Alaska, USA}
\begin{itemize}
\item Ice Arch construction team member
\item Assisted with the construction and setup of the wooden ice molds
\item Assisted with the transportation of water to form the arch
\item Assisted with the erection of the dual ice arch design
\end{itemize}

\section*{Honors and Awards}

\textbf{Chancellor’s List:} Spring 2017, Fall 2016, Spring 2016\\
\textbf{Dean’s List:} Fall 2015, Spring 2015\\
\textbf{University of Alaska Scholars Award:} Fall 2014 – Spring 2018\\
\textbf{Alaska Performance Scholarship:} Fall 2014 – Spring 2018\\
\textbf{NACE International Alaska Section / BP Scholarship:} Fall 2016 – Spring 2017\\
\textbf{Undergraduate Research and Scholarly Activity Award:} Spring 2018

\section*{Computer Skills}

\subsection*{Programming Languages}

Python, C/C++, MATLAB, Julia, Fortran

\subsection*{Platforms and APIs}

CUDA, Arduino, Raspberry Pi

\section*{Hobbies}

\subsection*{Martial Arts/Sports}

\textbf{Krav Maga Instructor} \textit{Winter 2017 – Winter 2018}\\
\textbf{Alaska Krav Maga \& Fitness}, \textit{Fairbanks, Alaska, USA}

Krav Maga, Muay Thai, Brazilian Jiu Jitsu, Hockey, Powerlifting

\section*{Personal Projects}

2W Blue LED Laser Driver, An Event in Fairbanks (Game), Minecraft Server

\section*{Other Interests}

Programming, Mathematics, Physics, Chemistry, Optics, Biomedical Engineering, Aerospace,
Artificial Intelligence, Quantum Communication, Game Development, Aviation, Music, Language
Learning

\section*{References}

\textbf{Javier Fochesatto PhD, Associate Professor of Atmospheric Sciences}\\
317 Akasofu Building, 930 Koyukuk Dr., University of Alaska, Fairbanks, Fairbanks, AK 99775\\
907-474-7602 $|$ gjfochesatto@alaska.edu

\textbf{Paul Bean, Aerospace Technology Engineer}\\
FOSS Lab, NASA Armstrong Flight Research Center, Edwards Air Force Base, CA 93523\\
661-276-2451 $|$ paul.bean@nasa.gov

\textbf{Cheng-fu Chen PhD, Professor of Mechanical Engineering}\\
Duckering 349D, 1760 Tanana Lp., University of Alaska, Fairbanks, Fairbanks, AK 99775\\
907-474-7265 $|$ cf.chen@alaska.edu

\end{document}
