%%%%%%%%%%%%%%%%%%%%%%%%%%%%%%%%%%%%%%%%%
% Twenty Seconds Resume/CV
% LaTeX Template
% Version 1.1 (8/1/17)
%
% This template has been downloaded from:
% http://www.LaTeXTemplates.com
%
% Original author:
% Carmine Spagnuolo (cspagnuolo@unisa.it) with major modifications by 
% Vel (vel@LaTeXTemplates.com)
%
% License:
% The MIT License (see included LICENSE file)
%
%%%%%%%%%%%%%%%%%%%%%%%%%%%%%%%%%%%%%%%%%

%----------------------------------------------------------------------------------------
%	PACKAGES AND OTHER DOCUMENT CONFIGURATIONS
%----------------------------------------------------------------------------------------

\documentclass[letterpaper]{twentysecondcv} % a4paper for A4

%----------------------------------------------------------------------------------------
%	 PERSONAL INFORMATION
%----------------------------------------------------------------------------------------

% If you don't need one or more of the below, just remove the content leaving the command, e.g. \cvnumberphone{}

\profilepic{AK_Researcher_logo_resume.png} % Profile picture

\cvname{Doug Keller} % Your name
\cvjobtitle{PhD Student} % Job title/career

%\cvdate{December 1994} % Date of birth
\cvaddress{France} % Short address/location, use \newline if more than 1 line is required
%\cvnumberphone{+33 06 41 04 75 18} % Phone number
\cvsite{www.alaskanresearcher.org} % Personal website
\cvmail{dg.kllr.jr@gmail.com} % Email address

%----------------------------------------------------------------------------------------

\begin{document}

%----------------------------------------------------------------------------------------
%	 ABOUT ME
%----------------------------------------------------------------------------------------

\aboutme{I'm a PhD student at the Laboratoire de Météorologie Dynamique in Palaiseau, France. I'm originally from Alaska and was raised in the small town of Chugiak. Growing up I played hockey competitively, peaking at the Junior A Tier III level in the American West Hockey League just before university. More recently, I've trained in Brazilian Jiu Jitsu, Krav Maga, and Muay Thai, and also taught the latter two. I tend to stay active, but I'm not afraid to relax and play a few video games either or even a game of chess. I'm a pretty open book, don't hesitate to send me an email asking for more information.} % To have no About Me section, just remove all the text and leave \aboutme{}

%----------------------------------------------------------------------------------------
%	 SKILLS
%----------------------------------------------------------------------------------------

% Skill bar section, each skill must have a value between 0 an 6 (float)
\skills{{Signal Processing/5},{Heat Transfer/5},{Fluid Mechanics/5.5},{Atmospheric Sciences/4.5},{Oceanography/4},{Python, C, CUDA, MATLAB, Fortran/4.5},{Optics/3}}

%------------------------------------------------

% Skill text section, each skill must have a value between 0 an 6
%\skillstext{{lovely/4},{narcissistic/3}}

%----------------------------------------------------------------------------------------

\makeprofile % Print the sidebar

%----------------------------------------------------------------------------------------
%	 INTERESTS
%----------------------------------------------------------------------------------------

\section{Interests}

Any and everything relating to the atmosphere and ocean, including numerical weather forecasting, optical phenomena, air-sea interaction, remote sensing, and more. I have experience in multiple areas of engineering and science. Parallel processing and GPU computation are also areas of interest.

%----------------------------------------------------------------------------------------
%	 EDUCATION
%----------------------------------------------------------------------------------------

\section{Education}

\begin{twenty} % Environment for a list with descriptions
	\twentyitem{since 2019}{Ph.D. Ingénierie, Mécanique et Énergétique}{École Polytechnique, IP Paris}{\emph{Impact of the spatial and temporal variability time of the Mistral on dense water formation in the Mediterranean Sea}}
	\twentyitem{2017-2018}{M.Sc. Mechanical Engineering}{University of Alaska Fairbanks}{\emph{Comparison Of Resistance-Based Strain Gauges And Fiber Bragg Gratings In The Presence Of Electromagnetic Interference Emitted From An Electric Motor}}
	\twentyitem{2014-2018}{B.Sc. Mechanical Engineering}{University of Alaska Fairbanks}{\emph{Magna Cum Laude w/ Aerospace Concentration}}
	%\twentyitem{<dates>}{<title>}{<location>}{<description>}
\end{twenty}

%----------------------------------------------------------------------------------------
%	 PUBLICATIONS
%----------------------------------------------------------------------------------------

\section{Publications}

\begin{twentyshort} % Environment for a short list with no descriptions
	\twentyitemshort{2019}{D. Keller, D. R. Eagan, G. J. Fochesatto, R. Peterson, \emph{Advantages of Fiber Bragg Gratings for Measuring Electric Motor Loadings in Aerospace Application} Review of Scientific Instruments}
	
%\subsection{\underline{Submitted}}
%\newline \\

	%\twentyitemshort{2020}{D. Keller, G. J. Fochesatto \emph{Seasonal Variation of Subarctic and Arctic Superior Mirages with GPSRO} Applied Optics}
\end{twentyshort}

%----------------------------------------------------------------------------------------
%	 AWARDS
%----------------------------------------------------------------------------------------

%\section{Awards}
%
%\begin{twentyshort} % Environment for a short list with no descriptions
%	\twentyitemshort{1987}{All-Time Best Fantasy Novel.}
%	\twentyitemshort{1998}{All-Time Best Fantasy Novel before 1990.}
%	%\twentyitemshort{<dates>}{<title/description>}
%\end{twentyshort}

%----------------------------------------------------------------------------------------
%	 EXPERIENCE
%----------------------------------------------------------------------------------------

\section{Experience}

\begin{twenty} % Environment for a list with descriptions
    \twentyitem{since 2019}{Ph.D. Thesis}{Laboratoire de Météorologie Dynamique}{Determining the spatial and temporal effects of the Mistral and Tramontane winds on the Northwestern Mediterranean Sea}
    \twentyitem{2020-2021}{Teaching Assistant}{École Polytechnique}{Taught both MEC559, Introduction to Wind Energy, and PHYS103, Bachelor's Physics Lab, for the Département de Mécanique.} 
    \twentyitem{2019}{Arctic and Subarctic Superior Mirages}{Geophysical Institute}{Determined the occurrence and variability superior mirages in the arctic and subarctic regions with GPS radio occultation.}
    \twentyitem{2019}{Research Technician}{Alaska Centery for Energy and Power}{Wrote the safety manual for the Energy Technology Facility and performed data analysis and organization for the Alaska Fuel Use Study.}
    \twentyitem{2017-2018}{M.Sc. Thesis}{College of Engineering and Mines}{Determined the effect of electromagnetic interference from electric motors on load sensing strain gauges utilizing fiber Bragg gratings.}
    \twentyitem{2017-2018}{Alaska Space Grant Fellowship}{Geophysical Institute}{Studied the atmospheric boundary layer with the use of NASA's MPLNET and developed a new wavelet.}
    \twentyitem{2017}{Mechanical Engineering Intern}{NASA Armstrong}{Tested the heat transfer capability of the Fiber Optic Sensing System housing for use on the X-59 X-plane}
    \twentyitem{2016}{Raman Spectroscopy Lidar}{Geophysical Institute}{Designed a beam splitter cube fixture for the optical layout of a H$_2$0 three phase detecting lidar.}
    %\twentyitem{<dates>}{<title>}{<location>}{<description>}
\end{twenty}

%----------------------------------------------------------------------------------------
%	 OTHER INFORMATION
%----------------------------------------------------------------------------------------

\section{References}

\begin{twenty}
    \twentyitem{Director}{Philippe Drobinski, Ph.D.}{Laboratoire de Météorologie Dynamique}{+33 01 69 33 51 42 \hspace{5pt} philippe.drobinski@lmd.polytechnique.fr}
    \twentyitem{Researcher}{Romain Pennel, Ph.D.}{Laboratoire de Météorologie Dynamique}{+33 01 69 33 52 33 \hspace{5pt} romain.pennel@lmd.polytechnique.fr}
    \twentyitem{Dept. Chair}{Javier Fochesatto, Ph.D.}{Atmospheric Sciences, Geophysical Institute}{+1 907 474 7265 \hspace{5pt} gjfochesatto@alaska.edu}
\end{twenty}

%----------------------------------------------------------------------------------------
%	 SECOND PAGE EXAMPLE
%----------------------------------------------------------------------------------------

%\newpage % Start a new page

%\makeprofile % Print the sidebar

%\section{Other information}

%\subsection{Review}

%Alice approaches Wonderland as an anthropologist, but maintains a strong sense of noblesse oblige that comes with her class status. She has confidence in her social position, education, and the Victorian virtue of good manners. Alice has a feeling of entitlement, particularly when comparing herself to Mabel, whom she declares has a ``poky little house," and no toys. Additionally, she flaunts her limited information base with anyone who will listen and becomes increasingly obsessed with the importance of good manners as she deals with the rude creatures of Wonderland. Alice maintains a superior attitude and behaves with solicitous indulgence toward those she believes are less privileged.

%\section{Other information}

%\subsection{Review}

%Alice approaches Wonderland as an anthropologist, but maintains a strong sense of noblesse oblige that comes with her class status. She has confidence in her social position, education, and the Victorian virtue of good manners. Alice has a feeling of entitlement, particularly when comparing herself to Mabel, whom she declares has a ``poky little house," and no toys. Additionally, she flaunts her limited information base with anyone who will listen and becomes increasingly obsessed with the importance of good manners as she deals with the rude creatures of Wonderland. Alice maintains a superior attitude and behaves with solicitous indulgence toward those she believes are less privileged.

%----------------------------------------------------------------------------------------

\end{document} 
