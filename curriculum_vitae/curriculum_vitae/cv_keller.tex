%Curriculum Vitae of Doug Keller

\documentclass[12pt,a4paper,sans]{moderncv}

\usepackage{hanging}

% moderncv themes
\moderncvstyle{classic}
\moderncvcolor{blue}

% adjust the page margins
\usepackage[a4paper, top = .5in, bottom = .55in, left = .75in, right = .75in]{geometry}

% personal data
\name{Douglas}{Keller Jr.}
\address{Gif-sur-Yvette, 91190, France}
\phone[mobile]{+33~06~41~04~75~18}
\email{dg.kllr.jr@gmail.com}  
\homepage{www.alaskanresearcher.org}
\social[linkedin]{dg-kllr-jr} 
\social[twitter]{AK\_Researcher}                

%----------------------------------------------------------------------------------
%            content
%----------------------------------------------------------------------------------

\begin{document}
%-----       resume       ---------------------------------------------------------
\makecvtitle
\vspace{-.75cm}

\section{Education}

\cventry{Fall 2019 -- Expected Fall 2022} {Doctorate of Philosophy: Mécanique des Fluides et des Solides, Acoustique}{École Polytechnique, Institut Polytechnique de Paris}{Palaiseau, Île-de-France, France}{}{}
\cventry{Fall 2017 -- Fall 2018}{Master of Science: Mechanical Engineering}{College of Engineering and Mines, University of Alaska Fairbanks}{Fairbanks, Alaska, USA}{}{GPA: 4.0/4.0}
\cventry{Fall 2014 -- Fall 2018}{Bachelor of Science: Mechanical Engineering}{College of Engineering and Mines, University of Alaska Fairbanks}{Fairbanks, Alaska, USA}{}{GPA: 3.8/4.0}

\section{Research Experience}

\cventry{Fall 2019 -- Present}{PhD Thesis}{École Polytechnique, Institut Polytechnique de Paris}{Palaiseau, France}{}{
Determining the spatial and temporal effects of the Mistral on the Dense Water Formation in the Gulf of Lion (Western Mediterranean Sea).\\
\textit{(In progress) Highlighted achievements:}
\begin{itemize}
\item Created a database of Mistral events with duration and intensity from atmospheric regional models driven by global weather reanalysis.
\item Ran multiple ocean simulations (NEMO) with different wind forcings to determine the ocean sensitivity to wind forcing.
\end{itemize}}

\cventry{Fall 2017 -- Fall 2018}{Master of Science Thesis}{University of Alaska Fairbanks}{Fairbanks, Alaska, USA}{}{
\textit{(In collaboration with and funded by NASA's Armstrong Fight Research Center)}\\
Determined the effect of electromagnetic interference from electric motors on load sensing strain gauges.\\
\textit{Highlighted achievements:}
\begin{itemize}
\item Placed NASA's Fiber Optic Sensing System (FOSS) and conventional strain gauge systems in high electromagnetic interference environments found in aviation
\item Developed a load cell designed around the FOSS to measure thrust from an electric motor coupled with a 3-blade propeller to develop future electric propulsion systems
\item Analyzed and compared data retrieved using the FOSS and conventional strain gauges
\item Determined the effect of electromagnetic interference on the thrust loading measurements
\item Determined the effect of mechanical interference/vibration with both systems in the same test application
\end{itemize}}

\cventry{Fall 2017 -- Summer 2018}{Alaska Space Grant Undergraduate Research Fellowship}{Geophysical Institute, University of Alaska Fairbanks}{Fairbanks, Alaska, USA}{}{
Studied the atmospheric boundary layer (ABL) with NASA's network of micro-pulse lidars (MPLNET).\\
\textit{Highlighted achievements:}
\begin{itemize}
\item Used the MPLNET and radiosonde data to determine the intraday variability of the high latitude atmospheric boundary layer
\item Developed algorithms to determine atmospheric boundary layer structural parameters from the MPLNET profiles
\item Compared and contrasted collocated lidar and radiosonde retrievals of ABL parameters
\item Produced statistical estimates of ABL parameters and computed diurnal and seasonal variations
\item Analyzed the high latitude ABL with the MPLNET, radiosonde data, and global positioning system radio occultation (GPSRO)
\end{itemize}}

\cventry{Fall 2016}{Raman Spectroscopy Lidar}{Geophysical Institute, University of Alaska Fairbanks}{Fairbanks, Alaska, USA}{}{
\textit{(Funded by the National Science Foundation)}\\
Assisted with the setup of a Raman spectroscopy lidar and developed mechanical fixtures for application.\\
\textit{Highlighted achievements:}
\begin{itemize}
\item Gained experience in optomechanical design for Lidar development
\item Developed simulation of Lidar signals to condition instrument design for 532 nm, N2, and H2O Raman vibrational bands
\item Implemented multichannel Lidar receiver
\end{itemize}}

\section{Vocational Experience}

\cventry{Summer 2019 -- Fall 2019}{Temporary Research Technician}{Alaska Center for Energy and Power, University of Alaska Fairbanks}{Fairbanks, Alaska, USA}{}{
Performing data analytics and large data organization for Alaska Center for Energy and Power's (ACEP) fuel meter project.\\
\textit{Highlighted achievements:}
\begin{itemize}
\item Developed large data analysis methods for the ACEP fuel meter project
\item Created data visualization methods for different temporal representations
\end{itemize}}

\cventry{Spring 2019 -- Summer 2019}{Temporary Research Technician}{Alaska Center for Energy and Power, University of Alaska Fairbanks}{Fairbanks, Alaska, USA}{}{
Rewriting the Alaska Center for Energy and Power's (ACEP) Energy Technology Facility's (ETF) safety manual.\\
\textit{Highlighted achievements:}
\begin{itemize}
\item Determining the job hazards and required training to safely perform experiments in the ETF
\item Adjusting the emergency action plan of ETF to more effectively transfer ETF's emergency systems
\end{itemize}}

\cventry{Spring 2017 -- Fall 2018}{Teaching Assistant}{College of Engineering and Mines, University of Alaska Fairbanks}{Fairbanks, Alaska, USA}{}{
Graded homework and exams, and assisted students with studies in engineering.\\
\textit{Highlighted achievements:}
\begin{itemize}
\item Graded homework and exams for the courses on Mechanics of Materials, Instrumentation and Measurement, and Heat Transfer
\item Recorded lectures for Astrodynamics and Mechanical Vibration
\end{itemize}}

\cventry{Summer 2017}{Mechanical Engineering Intern}{NASA Armstrong Flight Research Center}{Edwards, California, USA}{}{
Tested thermodynamics and heat transfer of the initial Fiber Optic Sensing System (FOSS) enclosure concept for the Quiet Supersonic Technology (QueSST) X-Plane (now the X-59).\\
\textit{Highlighted achievements:}
\begin{itemize}
\item Researched heat transfer technologies for FOSS enclosure flight testing application such as heat pipes and thermoelectric coolers
\item Analyzed heat transfer methods for FOSS enclosure application including foam insulation
\item Designed prototype enclosure for FOSS components, utilizing analyzed methods features foam insulation and heat pipes for improved high temperature environment survivability
\end{itemize}}

\cventry{Summer 2015}{Engineering Intern}{Alaska Department of Transportation}{Fairbanks, Alaska, USA}{}{
Determined the compliance of pedestrian facilities in the Fairbanks Borough with the American Disability Act (ADA).\\
\textit{Highlighted achievements:}
\begin{itemize}
\item Measured state maintained pedestrian facilities such as sidewalks and ramps
\item Determined pedestrian facility requirements for ADA compliance including required slope accommodation for disabled pedestrians
\item Compiled measurement data from multiple interns for coordinator review
\end{itemize}}

\section{Project Experience}

\cventry{Summer 2016}{Golf Swing Replicator Prototype}{Project Aisle}{Spokane, Washington, USA}{}{
Designed and built a golf swing replicating rig.\\
\textit{Highlighted achievements:}
\begin{itemize}
\item Designed the golf swing replicator prototype in SolidWorks 2014 with adjustable stand, ball holder, and swinging mechanism
\item Fabricated the golf swing replicator prototype, utilizing the mill, lathe, and MIG welder
\item Tested and analyzed golf swing replicator prototype
\end{itemize}}

\cventry{Fall 2014 -- Spring 2015}{Ice Arch Build}{University of Alaska Fairbanks}{Fairbanks, Alaska, USA}{}{
Part of the construction team that built the winning design for UAF's 2014 Ice Arch competition.\\
\textit{Highlighted achievements:}
\begin{itemize}
\item Assisted with the construction and setup of the wooden ice molds
\item Assisted with the transportation of water to form the arch
\item Assisted with the erection of the dual ice arch design
\end{itemize}}

\section{Publications}

\cvitem{}{\small\underline{D. Keller}, D. R. Eagan, G. J. Fochesatto, R. Peterson (2019) “Advantages of Fiber Bragg Gratings for Measuring Electric Motor Loadings in Aerospace Application.” \textit{Review of Scientific Instruments}}

\subsection{In Progress}

\cvitem{}{\small\underline{D. Keller}, G. J. Fochesatto. “A New Wavelet to Determine the Planetary Boundary Layer Height from Micro Pulse Lidar Backscatter.”}

\cvitem{}{\small\underline{D. Keller}, G. J. Fochesatto. "Identifying the Seasonal Conditions for the Occurrence of High Latitude Superior Mirages."}

\subsection{Conference Talks}

\cvitem{}{\small \underline{D. Keller}, G. J. Fochesatto. "Identifying the Conditions for the Occurrence of High Latitude Superior Mirages." \textit{Light and Color in Nature}, Bar Harbor, Maine, USA. July 15, 2019}

\subsection{Conference Proceedings}

\cvitem{}{\small G. J. Fochesatto, O. Galvez, P. Ristori, \underline{D. Keller}, and E. L. Fochesatto. “Lidar to Determine the Fractions of Ice, Liquid and Water Vapor in Polar Tropospheric Cloud.” \textit{Proceedings of the 28th International Laser Radar Conference}, Bucharest, Romania. 25-30 June 2017.}

\subsection{Conference Poster Presentations}

\cvitem{}{\small\underline{D. Keller}, G. J. Fochesatto. “RAOBs and Micro Pulse Lidar Determination of the Atmospheric Boundary Layer.” \textit{Alaska Space Grant Program Annual Symposium}, Anchorage, Alaska, USA. April 20, 2018.}

\cvitem{}{\small\underline{D. Keller}, D. Eagan. “FOSS Load Cell.” \textit{Undergraduate Research and Scholarly Activity Research Day}, Fairbanks, Alaska, USA. April 10, 2018.}

\cvitem{}{\small\underline{D. Keller}, G. J. Fochesatto, Ellsworth Welton, Jasper Lewis, James Campbell, and Sebastian Stewart. "Methodology for PBL Retrieval Based on NASA MPLNET Datasets." \textit{99th Annual American Meteorological Society Meeting}, Phoenix, Arizona, USA, Jan. 6-10, 2019.}

\section{Programming}

\cvitem{Languages}{Python, C/C++, MATLAB, Julia, Fortran}{}{}
\cvitem{Platforms and APIs}{Linux, Windows, CUDA, Arduino, Raspberry Pi}{}{}

%\section{Honors and Awards}
%
%\cvitem{Spring 2018}{Undergraduate Research and Scholarly Activity Award}
%\cvitem{Fall 2014 – Spring 2018}{University of Alaska Scholars Award}
%\cvitem{Fall 2014 – Spring 2018}{Alaska Performance Scholarship}
%\cvitem{Fall 2016 – Spring 2017}{NACE International Alaska Section / BP Scholarship}
%\cvitem{Spring 2017, Fall 2016, Spring 2016}{Chancellor’s List}
%\cvitem{Fall 2015, Spring 2015}{Dean’s List}

\section{Sports}

\cventry{Winter 2017 -- Winter 2018}{Krav Maga Instructor}{Alaska Krav Maga \& Fitness}{Fairbanks, Alaska, USA}{}{
Taught Krav Maga, Muay Thai, and fitness classes to students at varying levels of skill. Also trained in Brazilian Jiu Jitsu.}

\cvitem{Hockey}{
Played hockey competitively until 2014, peaking at the Junior A Tier III level in the AWHL (now part of the NA3HL). Played intramurals and beer league from then on.}
\cvitem{Skiing}{
Skiing downhill and backcountry (with skins and the whole setup for charging steeper lines).}

\section{Hobbies}

\cvitem{Other Interests}{Aerospace and aviation, game development, optics and physics, artificial intelligence, quantum communication, music, language learning.}

\section{References}

\cvitem{}{\textbf{Javier Fochesatto PhD, Associate Professor of Atmospheric Sciences} \newline
317 Akasofu Building, 930 Koyukuk Dr., University of Alaska Fairbanks, Fairbanks, AK 99775 \newline 
907-474-7602 $|$ gjfochesatto@alaska.edu}

\cvitem{}{\textbf{Dayne Broderson, Computer and Information Research Scientist} \newline
1764 Tanana Lp., University of Alaska Fairbanks, Fairbanks, AK 99775 \newline
broderson@gmail.com}

\cvitem{}{\textbf{Paul Bean, Aerospace Technology Engineer} \newline
FOSS Lab, NASA Armstrong Flight Research Center, Edwards Air Force Base, CA 93523 \newline
661-276-2451 $|$ paul.bean@nasa.gov}

\cvitem{}{\textbf{Cheng-fu Chen PhD, Professor of Mechanical Engineering} \newline
Duckering 349D, 1760 Tanana Lp., University of Alaska Fairbanks, Fairbanks, AK 99775 \newline
907-474-7265 $|$ cf.chen@alaska.edu}

\end{document}
